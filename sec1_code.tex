\section{Code}

  \subsection{Python}

        \begin{lstlisting}[language=Python, caption={Python code to create an incidence matrix}]
        import numpy as np
            
        def incmatrix(genl1,genl2):
            m = len(genl1)
            n = len(genl2)
            M = None #to become the incidence matrix
            VT = np.zeros((n*m,1), int)  #dummy variable
            x = 0
            #compute the bitwise xor matrix
            M1 = bitxormatrix(genl1)
            M2 = np.triu(bitxormatrix(genl2),1) 
        
            for i in range(m-1):
                for j in range(i+1, m):
                    [r,c] = np.where(M2 == M1[i,j])
                    for k in range(len(r)):
                        VT[(i)*n + r[k]] = 1;
                        VT[(i)*n + c[k]] = 1;
                        VT[(j)*n + r[k]] = 1;
                        VT[(j)*n + c[k]] = 1;
                        
                        if M is None:
                            M = np.copy(VT)
                        else:
                            M = np.concatenate((M, VT), 1)
                        
                        VT = np.zeros((n*m,1), int)
            
            return M



    
        \end{lstlisting}
\newpage

\subsection{Matlab}


\begin{lstlisting}[language=Matlab, caption={MATLAB Code that tests Solar System Function}]
    function test_advanced_level(unit_under_test)
    % TEST_ADVANCED_LEVEL Test the simulator against the advanced level of achievement.
    %
    % TEST_ADVANCED_LEVEL(@unit) tests a function called "unit" instead of the
    % default, "solarsystem".
    %
    % This is provided as a means for your to test your program's accuracy. We
    % supply here high precision answers that can use as a benchmark against
    % which to compare your code.
    %
    % A program similar to this one will be used during marking to test your
    % program's accuracy and speed. We'll use different initial conditions, so
    % don't try simply hard-coding these answers! :)
    %
    
    % Default to a function named "solarsystem"
    if nargin < 1
        unit_under_test = @solarsystem;
    end
    
    % Data
    % Data source: NASA JPL Development Emphemeris DE405, imported into Matlab
    % using https://au.mathworks.com/matlabcentral/fileexchange/46074-jpl-ephemeris-manager
    mass = [1.98879724324801e+30;3.30167548185139e+23;4.86825414184162e+24;5.97333182929537e+24;6.41814989746695e+23;1.89888757501372e+27;5.68569250232054e+26;8.68357411676561e+25;1.02450682828011e+26;1.47100387814202e+22];
    p = [-410978934.937975 -52564098.573049 -11647539.5911275;-20263704896.5463 37298969437.5484 21998926177.1807;107457059203.846 12751258164.7855 -1081247256.91775;-104473131433.549 95807463843.1787 41554965796.5625;-47532402438.2755 -197479402904.819 -89286739068.5338;740812325977.265 -29623952257.2314 -30753799138.017;-391719672964.493 1189107854643.27 507856891148.711;-2396814857836.84 -1270773906334.37 -522608874439.045;-1545201887440.28 -3957617757444.78 -1581427940931.15;-4371341308972.33 -1084064015240.84 978703610774.062];
    v = [1.94673233456669 -10.8814016462929 -4.7775329435922;-54017.2779417951 -18415.0969798133 -4228.50548119061;-3793.57777814318 31524.0648690534 14419.9306824639;-21597.9402281813 -19392.9951239518 -8410.50277824797;24596.1594690375 -2563.11636886769 -1841.7251251432;538.777252737696 12558.0983493514 5370.16231719295;-9767.15104601119 -2764.87492216388 -721.832483731844;3335.76872430951 -5686.29309895411 -2537.72389267233;5074.99185394443 -1640.69964089467 -797.853610190395;1586.81468930053 -5301.34210829372 -2132.29213550457];
    
    % Use inner planets only; supply them in the order Sun, Earth, Mercury,
    % Venus, Mars (so that colours used for the Sun and Earth in the other
    % tests apply here too)
    i = [1 4 2 3 5];
    mass = mass(i);
    p = p(i,:);
    v = v(i,:);
    
    % Test 1
    Test_3D_Solar_System(false);
    
    % Test 2
    Test_3D_Solar_System(true);
    
        function test_result(parameter, value, units, comparator, benchmark)
            if strcmp(units, '%')
                fprintf('  %28s :  %-15.6f', [parameter ' (' units ')'], value);
            else
                fprintf('  %28s :  %-15.6g', [parameter ' (' units ')'], value);
            end
    
            if nargin == 5
                if comparator(value, benchmark)
                    fprintf('   ** PASS. Meets or exceeds the expectation of %g%s', benchmark, units);
                else
                    fprintf('   ** FAIL. Does not meet the expectation of %g%s', benchmark, units);
                end
            end
            fprintf('\n');
        end
    
    
        function Test_3D_Solar_System(speed_test)
            if speed_test
                fprintf('<strong>*** [Advanced level] Inner planets in 3D (execution speed test)</strong>\n');
            else
                fprintf('<strong>*** [Advanced level] Inner planets in 3D</strong>\n');
            end
    
            % Run the program
            tic();
            [final_p, final_v] = unit_under_test(p, v, mass, 400*24*60*60, speed_test);
            t = toc();
            test_result('Execution time', t, 's');
            
            % Check the dimensions of the return values
            assert(all(size(final_p) == size(p)), 'Expected size of return value "p" to be %ix%i, received %ix%i instead.', size(p,1), size(p,2), size(final_p, 1), size(final_p, 2));
            assert(all(size(final_v) == size(v)), 'Expected size of return value "v" to be %ix%i, received %ix%i instead.', size(v,1), size(v,2), size(final_p, 1), size(final_p, 2));
            
            % Check the answers
            correct_p = [-345966512.946938 -427734515.389726 -176305011.39972;-146624134718.92 23657267681.4718 10263838966.7501;18103318019.8 -57037628697.2036 -32327489902.5598;32774480992.525 -94215943100.9452 -44459425421.2662;-136512481872.172 183605819453.111 87921613716.9978];
            correct_v = [1.80779317569275 -10.8097599943337 -4.73922210995349;-5754.8524496641 -27006.432527836 -11711.495344779;37086.6176187111 15367.1705312566 4360.22910247245;33125.9176644671 10361.9441289187 2564.05625302394;-19228.3713286498 -10562.8647670158 -4323.54314113474];
            
            % mercury is harder to simulate because it moves the fastest
            % the objects are in this order: Sun, Earth, Mercury, Venus, Mars
            expectations = [0.1 1 5 1 1];
            for i = 1:size(p,1)
                test_result(sprintf('Object %i position error', i), norm(final_p(i,:) - correct_p(i,:))/norm(correct_p(i,:))*100, '%', @le, expectations(i));
                test_result(sprintf('Object %i velocity error', i), norm(final_v(i,:) - correct_v(i,:))/norm(correct_v(i,:))*100, '%', @le, expectations(i));
            end  
        end
    
    end
\end{lstlisting}
    