\section{Feature Demonstration}

\subsection{Images}

I made a couple of latex commands that make inserting images a bit shorter. I made a command called \code{\insertimage{}{}{}}, which takes three arguments: the image filename, the caption, and the label. I also made a command called \code{\insertbigimage{}{}{}}, which does the same thing, but for big images that are too wide to be placed in a single column. \Cref{normal_image,big_image} are the result of using these commands (Though the big image is too big to fit on this page).

\insertimage{c3.png}{Example Image}{normal_image}

\insertbigimage{R6.pdf}{Example Big Image (Made using MATLAB)}{big_image}

\subsection{Math}

This isn't really a feature of my template, but more a feature of \LaTeX \ itself. 
Here is some Math:
\begin{equation}
    C = A \times B = \begin{bmatrix}
        c_{11} & c_{12} & c_{13} \\
        c_{21} & c_{22} & c_{23} \\
        c_{31} & c_{32} & c_{33}
    \end{bmatrix}
\end{equation}
Here is some more ChatGPT gave me:
\begin{equation}
    \hat{f}(\xi) = \int_{-\infty}^{\infty} f(x) e^{-2 \pi i x \xi} \, dx
\end{equation}
I made a couple of commands that make writing math a bit easier. For example, I made a command called \code{\E{}}, which takes one argument, the exponent. For example, \code{\E{3}} will display as $\E{3}$. I also made a command called \code{\abs{}}, which takes one argument, the value to be enclosed in absolute value bars. For example, \code{\abs{-3}} will display as $\abs{-3}$ \footnote[1]{Only works in a math environment, either in an align/equation environment or between \$ \$ symbols. By the way footnotes are also a thing in \LaTeX.}


\subsection{Tables}

I don't have any custom commands to make tables, but I can still make them My recommendation is to use ChatGPT to generate the table for you.
\begin{table}[h!]
    \centering
    \begin{tabular}{ll}
        \toprule
        \textbf{Column 1} & \textbf{Column 2} \\
        \midrule
        Data 1 & Data 2 \\
        More Data 1 & More Data 2 \\
        Even More Data 1 & Even More Data 2 \\
        Dummy Data 1 & Dummy Data 2 \\
        Another Data 1 & Another Data 2 \\
        \bottomrule
    \end{tabular}
    \caption{A small table}
    \label{table_dummy}
\end{table}

I (mostly ChatGPT) have configured a way for latex to read \code{.xlsx} and \code{.csv} files, which can be seen in \cref{table_csv}. This table was generated using a \code{.csv} file, with the column names changed to have nice math in them. \Cref{table_csv} shows this

\begin{table}[p]
    \centering
    \begin{tabular}{ccccc} % Adjust the widths as needed
        \toprule
        \begin{tabular}[c]{@{}c@{}}Reco-\\rding\end{tabular} & 
        \begin{tabular}[c]{@{}c@{}}$\Delta P_{\text{avg}}$ \\$ - \Delta P_{\text{avg}}'$ (m)\end{tabular} &
        \begin{tabular}[c]{@{}c@{}}$\Delta P_{\text{fin}}$ \\$ - \Delta P_{\text{fin}}'$ (m)\end{tabular} & 
        \begin{tabular}[c]{@{}c@{}}$\Delta P_{\text{max}}$ \\$ - \Delta P_{\text{max}}'$ (m)\end{tabular} \\
        \midrule
        \csvreader[head to column names, late after line=\\]{tables/data.csv}{}
        {\csvcoli & \csvcolii & \csvcoliii & \csvcoliv}
        \bottomrule
    \end{tabular}
    \caption{Data from \code{tables/data.csv}}
    \label{table_csv}
\end{table}

\subsection{Code}

I also made a command for inline code, \code{\code{}}. This command takes one argument, the code to be displayed. For example, \code{\code{print("Hello World")}} will display as \code{print("Hello World")}. Groundbreaking, I know 


